\documentclass[a4paper, 12pt]{book}
\usepackage{comment} 
\usepackage{fullpage}
\usepackage[utf8x]{inputenc}
\usepackage[T2A]{fontenc}
\usepackage[english,russian]{babel}
\pagestyle{empty}

\setcounter{chapter}{2}
\setcounter{section}{3}

\setcounter{subsection}{2}
\setcounter{equation}{16}

\usepackage[fleqn]{amsmath}

\usepackage{color}
\definecolor{light-gray}{rgb}{0.8,0.8,0.8}

\begin{document}
\noindent
\large\textit{2.4 Fluid mechanics: Drag} \hfill \textbf{25} \\
\begin{equation}\frac{\mathcal{V}^2}{gh}=dimensionless \ constant.\end{equation} 
(The right side is a dimensionless constant because no second group is
available to use there.) In other words, $\mathcal{V}^2/gh $ or $\mathcal{V}\sim \sqrt{lg}$.

This result reproduces the result of the less sophisticated dimensional
analysis in Section 1.2. Indeed, with only one dimensionless group, either
analysis leads to the same conclusion. However, in hard problems—for
example, finding the drag force—the less sophisticated method does not
provide its constraint in a useful form; then the method of dimensionless
groups is essential.\\

\colorbox{light-gray}{
\begin{minipage}{\textwidth}
\textbf {Problem 2.15 Fall time}

Synthesize an approximate formula for the free-fall time t from g and h.

\textbf {Problem 2.16 Kepler’s third law}

Synthesize Kepler’s third law connecting the orbital period of a planet to its
orbital radius. (See also Problem 1.15.)

\end{minipage}
} 

\textit{What dimensionless groups can be constructed for the drag problem?}\\

One dimensionless group could be $F/\rho\mathcal{V}^2r^2$; a second group could be $rv/v$.
Any other group can be constructed from these groups (Problem 2.17), so
the problem is described by two independent dimensionless groups. The
most general dimensionless statement is then
\begin{equation}one group = f(second group),\end{equation}
where f is a still-unknown (but dimensionless) function.\\

\textit{Which dimensionless group belongs on the left side?}\\

The goal is to synthesize a formula for F, and F appears only in the first
group $F/\rho\mathcal{V}^2r^2$. With that constraint in mind, place the first group on the
left side rather than wrapping it in the still-mysterious function f. With
this choice, the most general statement about drag force is
\begin{equation}\frac{F}{\rho\mathcal{V}^2r^2}=f\left(\frac{r\mathcal{V}}{v}\right).\end{equation}
The physics of the (steady-state) drag force on the cone is all contained
in the dimensionless function f.
\newpage
\large\textbf{26} \hfill \textit{2 Easy cases} \\
\colorbox{light-gray}{
\begin{minipage}{\textwidth}
\textbf {Problem 2.17 Only two groups}

Show that F, $\mathcal{V}$, r, $\rho$, and v produce only two independent dimensionless groups.

\textbf {Problem 2.18 How many groups in general?}

Is there a general method to predict the number of independent dimensionless
groups? (The answer was given in 1914 by Buckingham [9].)
\end{minipage}}
The procedure might seem pointless, having produced a drag force that
depends on the unknown function f. But it has greatly improved our
chances of finding f. The original problem formulation required guessing
the four-variable function h in $F = h(\mathcal{V}, r, p, v)$, whereas dimensional
analysis reduced the problem to guessing a function of only one variable
(the ratio $\mathcal{V}r/v$). The value of this simplification was eloquently described
by the statistician and physicist Harold Jeffreys (quoted in [34, p. 82]):

\normalsize{A good table of functions of one variable may require a page; that of a function
of two variables a volume; that of a function of three variables a bookcase;
and that of a function of four variables a library.}

\large
\colorbox{light-gray}{
\begin{minipage}{\textwidth}
\textbf {Problem 2.19 Dimensionless groups for the truncated pyramid}

Synthesize an approximate formula for the free-fall time t from g and h.

\textbf {Problem 2.16 Kepler’s third law}

The truncated pyramid of Section 2.3 has volume
\begin{equation}V=\frac{1}{3}=\left(a^2 + ab + b^2\right).\end{equation}
Make dimensionless groups from V, h, a, and b, and rewrite the volume using
these groups. (There are many ways to do so.)
\end{minipage}
} 

\textbf {2.4.2 Using easy cases}
Although improved, our chances do not look high: Even the one-variable
drag problem has no exact solution. But it might have exact solutions in
its easy cases. Because the easiest cases are often extreme cases, look first
at the extreme cases.\\

\textit{Extreme cases of what?}\\

The unknown function f depends on only $r\mathcal{V}/v$,
\begin{equation}\frac{F}{pv^2r^2}=f\left(\frac{rv}{v}\right),\end{equation}
so try extremes of rv/v. However, to avoid lapsing into mindless symbol
pushing, first determine the meaning of $r\mathcal{V}/v$. This combination $r\mathcal{V}/v$,
\\
\large\textit{2.4 Fluid mechanics: Drag} \hfill \textbf{27} \\

often denoted Re, is the famous Reynolds number. (Its physical interpretation
requires the technique of lumping and is explained in Section 3.4.3.)
The Reynolds number affects the drag force via the unknown function f:
\begin{equation}\frac{F}{p\mathcal{V}^2r^2}=f\left(\frac{r\mathcal{V}}{v}\right),\end{equation}
With luck, f can be deduced at extremes of the Reynolds number; with
further luck, the falling cones are an example of one extreme.

\textit{Are the falling cones an extreme of the Reynolds number?}

The Reynolds number depends on r,$\mathcal{V}$, and v. For the speed $\mathcal{V}$, everyday
experience suggests that the cones fall at roughly 1 m $s^1$ (within, say, a
factor of 2). The size r is roughly 0.1 m (again within a factor of 2). And
the kinematic viscosity of air is $\mathcal{V}\sim 10^{−5}m^{2}s^{−1}$. The Reynolds number is
\begin{equation}\frac{\overbrace{0.1m}^r\times \overbrace{1m^{-1}}^\mathcal{V}}{\underbrace{10^{−5}m^{2}s^{−1}}_\mathcal{V}}\sim 10^4,\end{equation}
It is significantly greater than 1, so the falling cones are an extreme case of high Reynolds number. (For low Reynolds number, try Problem 2.27 and see [38].)

\colorbox{light-gray}{
\begin{minipage}{\textwidth}
\small{\textbf {Problem 2.19 Dimensionless groups for the truncated pyramid}

Synthesize an approximate formula for the free-fall time t from g and h.

\textbf {Problem 2.20 Reynolds numbers in everyday flows}

Estimate Re for a submarine cruising underwater, a falling pollen grain, a falling raindrop, and a 747 crossing the Atlantic.}

\end{minipage}
}
The high-Reynolds-number limit can be reached many ways. One way is to shrink the viscosity v to 0, because v lives in the denominator of the Reynolds number. Therefore, in the limit of high Reynolds number, viscosity disappears from the problem and the drag force should not de- pend on viscosity. This reasoning contains several subtle untruths, yet its conclusion is mostly correct. (Clarifying the subtleties required two cen- turies of progress in mathematics, culminating in singular perturbations and the theory of boundary layers [12, 46].)
Viscosity affects the drag force only through the Reynolds number:
\begin{equation}\frac{F}{p\mathcal{V}^2r^2}=f\left(\frac{r\mathcal{V}}{v}\right).\end{equation}
\newpage
\large\textbf{28} \hfill \textit{2 Easy cases} \\
To make F independent of viscosity, F must be independent of Reynolds number! The problem then contains only one independent dimensionless group, F/pv2r2, so the most general statement about drag is
\begin{equation}\frac{\mathcal{V}^2}{gh}=dimensionless \ constant.\end{equation} 
The drag force itself is then $F \sim p\mathcal{V}^2r^2$. Because $r^2$ is proportional to the cone’s cross-sectional area A, the drag force is commonly written
\begin{equation}F=\sim p\mathcal{V}^2A.\end{equation} 
Although the derivation was for falling cones, the result applies to any object as long as the Reynolds number is high. The shape affects only the missing dimensionless constant. For a sphere, it is roughly 1/4; for a long cylinder moving perpendicular to its axis, it is roughly 1/2; and for a flat plate moving perpendicular to its face, it is roughly 1.
\section{Terminal velocities}
The result $F\sim p\mathcal{V}^2A$ is enough to predict the terminal veloci- ties of the cones. Terminal velocity means zero acceleration, so the drag force must balance the weight. The weight is $W = \sigma_{pape}rA\sigma_{paper}g$, where paper is the areal density of paper (mass per area) and $A\sigma_{paper}$ is the area of the template after cutting out the quarter section. Because $A\sigma_{paper}$ is comparable to the cross-sectional area A, the weight is roughly given by
\begin{equation}W\sim \sigma_{paper} Ag\end{equation} 
Therefore
\begin{equation} \underbrace{p\mathcal{V}^2 A }_{drag}\sim \underbrace{\sigma_{paper} Ag }_{weight}\end{equation}
The area divides out and the terminal velocity becomes
\begin{equation}
V \sim \sqrt {\frac{\sigma_{paper} g}{p}}.
\end{equation}
All cones constructed from the same paper and having the same shape, $whatever\ their\ size$, fall at the same speed!
\end{document}







